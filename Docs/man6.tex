\documentclass[12pt]{article}
\pagestyle{plain}
\setlength {\topmargin} {-1.0cm}
\setlength {\textwidth} {16.0cm}
\setlength {\textheight} {22.5cm}
\setlength {\oddsidemargin} {0cm}
\begin{document}

%  Useful newcommands
%  @@@@@@@@@@@@@@@@@@

\newcommand{\etaI}{\eta_{\scriptscriptstyle I}}
\newcommand{\etaR}{\eta_{\scriptscriptstyle R}}
\newcommand{\etaU}{\eta_{\scriptscriptstyle U}}
\newcommand{\BH}{\scriptscriptstyle BH}
\newcommand{\GR}{\scriptscriptstyle GR}
\newcommand{\barrho}{\mathop{<\mskip -4mu\rho\mskip -4mu>}}
\newcommand{\brho}{\mbox{\boldmath $\rho $}} %due to Sue Cowell 9/99
\newcommand{\bom}{\mbox{\boldmath $\omega $}}
\newcommand{\bsig}{\mbox{\boldmath $\sigma $}}
\newcommand\pminus{\phantom{-}}
\newcommand{\etal}{{\it et al.\ }}
\newcommand{\cm}{c.m.\ }
\newcommand{\ie}{i.e.\ }
\newcommand{\cf}{cf.\ }
\newcommand{\rn}[1]{(\ref{#1})}
\newcommand{\be}{\begin{equation}}
\newcommand{\ee}{\end{equation}}
\newcommand{\bea}{\begin{eqnarray}}
\newcommand{\eea}{\end{eqnarray}}
\newcommand{\bean}{\begin{eqnarray*}}
\newcommand{\eean}{\end{eqnarray*}}
\newcommand{\bdis}{\begin{displaymath}}
\newcommand{\edis}{\end{displaymath}}
\newcommand{\half}{{\textstyle \frac{1}{2}}}
\newcommand{\third}{{\textstyle \frac{1}{3}}}
\newcommand{\quart}{{\textstyle \frac{1}{4}}}
\newcommand{\sixth}{{\textstyle \frac{1}{6}}}
\newcommand{\tenth}{{\textstyle \frac{1}{10}}}
\newcommand{\twelfth}{{\textstyle \frac{1}{12}}}
\newcommand{\sixteenth}{{\textstyle \frac{1}{16}}}
\newcommand{\twothird}{{\textstyle \frac{2}{3}}}
\newcommand{\eigth}{{\textstyle \frac{1}{8}}}
\newcommand{\onetfour}{{\textstyle \frac{1}{24}}}
\newcommand{\onehtw}{{\textstyle \frac{1}{120}}}
\newcommand{\thalf}{{\textstyle \frac{3}{2}}}
\newcommand{\fhalf}{{\textstyle \frac{5}{2}}}
\newcommand{\fthird}{{\textstyle \frac{5}{3}}}
\newcommand{\fat}[1]{{\bf #1}}
\newcommand{\doo}[2]{{\frac{\partial #1}{\partial #2}}}
\newcommand{\ex}{{\rm e}}
\newcommand{\bc}{\begin{center}}
\newcommand{\ec}{\end{center}}
\newcommand{\hb}{\hfill\break}
\newcommand{\bi}{\begin{itemize}}
\newcommand{\ei}{\end{itemize}}
\newcommand{\kms}{{\rm km}\,{\rm s}^{-1}}
\newcommand{\pcc}{{\rm pc}^{-3}}
\newcommand{\bt}{\begin{tabbing}}
\newcommand{\et}{\end{tabbing}}
\newcommand{\vinf}{v_\infty}
\newcommand{\kmsec}{km\,${\rm s}^{-1}$}
\newcommand{\kmsecc}{km\,${\rm s}^{-1}\:$}
\def\ZZ#1{$\scriptstyle #1$}
\def\plusplus{\raise 0.3ex\hbox{${\scriptstyle ++}$}{}}

\def\AJ{{\it Astron. J.} }
\def\APJ{{\it Astrophys. J.} }
\def\ApJ{{\it Astrophys. J.} }
\def\MN{{\it Mon. Not. R. Astron. Soc.} }
\def\AA{{\it Astron. Astrophys.} }
\def\CM{{\it Celes. Mech.} }
\def\CMD{{\it Celes. Mech. Dyn. Ast.} }
\def\PASJ{{\it Publ. Astron. Soc. Japan} }
\def\SJA{S.J. Aarseth}
\def\SAA{S.J. Aarseth }

%%%%%%%%%%%%%%%%%%%%%%%%%%%%%%%%%%%%%%%%%%%%%
% Alison's definitions of bold \rho and \mu (three subscript levels)

\font\elevenmib=cmmib10 at 11pt \skewchar\elevenmib='177
\font\eightmib=cmmib10 at 8pt   \skewchar\eightmib='177
\font\sixmib=cmmib10 at 6pt     \skewchar\sixmib='177
%
\newfam\mibfam
\textfont\mibfam=\elevenmib
\scriptfont\mibfam=\eightmib
\scriptscriptfont\mibfam=\sixmib

\def\hexnum#1{\ifnum#1<10 \number#1\else
 \ifnum#1=10 A\else\ifnum#1=11 B\else\ifnum#1=12 C\else
 \ifnum#1=13 D\else\ifnum#1=14 E\else\ifnum#1=15 F\fi\fi\fi\fi\fi\fi\fi}
\def\mib{\hexnum\mibfam}

\mathchardef\bmu="0\mib16

%\setlength {\topmargin} {-1.0cm}
%\setlength {\textwidth} {16.0cm}
%\setlength {\textheight} {22.5cm}
 
 
\centerline {\Large {\bf {NBODY6 User Manual~~~~}}}
\bigskip
\centerline {\Large {Sverre Aarseth~~~}}
\bigskip
\centerline {\large {\tt Email: sverre@ast.cam.ac.uk}~~~~~}
\bigskip
\centerline {\large {Institute of Astronomy, University of Cambridge~~~~}}
\medskip
\bigskip
\section{Introduction}

This is a User Manual for {\ZZ{NBODY6}}, a code that has been extensively
tested since it was first developed around 1992.
This code is mainly intended for laptops and workstations with the {\tt UNIX}
operating system.
A parallel implementation called {\ZZ{NBODY6\plusplus}} for Beowulf PC
clusters and the T3E supercomputer is also available [Spurzem \etal 2003].

The code relies on many features of the classical {\ZZ{NBODY5}} which dates
back to the late 1970s.
The subsequent change from a divided difference formulation to Hermite
integration [Makino 1991] led to a complete revision.
Briefly stated, single particles and centre-of-mass systems are integrated
by the Ahmad--Cohen [1973] neighbour scheme using the fourth-order Hermite
method [Makino \& Aarseth 1992] (for the quantized version see Aarseth 1999).
Binaries and close two-body encounters are studied by the Stumpff version of
Kustaanheimo--Stiefel [1964, KS] regularization [Mikkola \& Aarseth 1998],
while interactions of compact subsystems are described by the chain
regularization method [Mikkola \& Aarseth 1990, 1993, 1996].
Moreover, strong interactions in unperturbed triples and quadruples are
treated by three-body [Aarseth \& Zare 1974, AZ] and Heggie [1974] global
regularization [Mikkola 1985].
Finally, hard triples and higher-order systems satisfying a stability
criterion [Mardling \& Aarseth 1999, Mardling 2008] are reduced to two-body
configurations (so-called mergers as opposed to collisions).
All the relevant equations of motion are derived and discussed extensively in
a recent book [Aarseth 2003], together with algorithms which may be helpful
when examining the {\ZZ{FORTRAN}} procedures.
Hence familiarity with the book is beneficial for understanding the code.

Several aspects of synthetic stellar evolution, such as mass loss, tidal
circularization and collisions have been incorporated while still maintaining
energy conservation by means of appropriate correction procedures.
The early scheme used fast fitting functions for the radii and luminosities of
single stars of solar metallicity [Eggleton, Fitchett \& Tout 1989], with
imposed wind and supernova mass loss.
Binary evolution and collisions were subsequently included [Tout \etal 1997].
Finally, an extension to lower metallicities was implemented
[Hurley, Pols \& Tout 2000].
Note that the private {\ZZ{GRAPE-6}} code {\ZZ{NBODY4}}
contains several important astrophysical processes which are now also included
in the November 2007 release ({\it e.g.} Roche lobe overflow and spin--orbit
coupling).
Moreover, {\ZZ{NBODY6}} contains new procedures for a general 3D galactic
potential.
The code itself can be downloaded from the web in the form of a compressed
tar file.\footnote{\tt http://www.ast.cam.ac.uk/research.nbody.}

\section{Code structure}

The whole code consists of some 48,500 lines including comments and layout
space.
It is written in {\ZZ{FORTRAN}} and is F77 compliant but also compiles
with Intel or F95 (ignore warning messages of non-standard expressions).
There are about 296 routines altogether, mostly with mnemonic names of
maximum six characters.
Likewise, almost all the {\ZZ{FORTRAN}} statements are in upper case while
the comments are in lower case.
The coding generally conforms to a strict style and layout for clarity.

The main code relies heavily on a general {\tt common} block called
{\tt common6.h} which contains a large number of arrays (mostly size $N$) and
many useful scalars.
This enables a calculation to be split into several parts by saving all the
{\tt common} variables after a specified {\tt CPU} time (or by a
`{\tt touch STOP}' facility at arbitrary times, see below), followed by a
restart.
Except for some special situations, this gives rise to reproducible results
which are essential for experimental purposes as well as diagnostic
investigation.

The code consists of three main parts: {\ZZ {input}}, {\ZZ {output}} and
{\ZZ {integration}}, with the latter split into several large routines
employing different methods.
Thus a smooth running relies on treating a range of special cases by the
appropriate algorithm.
However, the decision-making requires very little overheads.
A number of optional procedures are included but care is needed to avoid
mutual inconsistencies since only some parameter values and no options are
checked.
The main choice is between isolated point-mass calculations and
realistic star cluster models which include relevant astrophysical processes.

\section{Getting started}

Once the file {\tt nbody6.tar.gz} is downloaded and uncompressed, the routines
are extracted by `{\tt tar xvf nbody6.tar}' and copied to four directories
for convenience: {\tt Ncode, Chain, Nchain, Docs}.
The first contains the main code, while {\tt Chain} holds the basic chain
procedures and {\tt Nchain} deals with the corresponding $N$-body interface.
A number of useful files, such as test input templates, listing in {\ZZ {TeX}}
of all routines and the structure of the old code {\ZZ {NBODY5}} are included
in the directory {\tt Docs}.

The size of most large {\tt common} arrays are given by the parameter file
{\tt params.h} and defined in Table~\ref{params}.
Depending on available memory, it is recommended to limit the maximum array
sizes somewhat but still leave room for bigger calculations; this will
facilitate examining {\tt common} blocks using the same {\tt read}
statements for different memberships.
Note that the files {\tt params.h} and {\tt common6.h} are also used in the
interface directory {\tt Nchain} by a soft link definition.
Once these parameters have been specified, the correct size of the
{\tt common} block is calculated automatically in routine {\tt mydump.f}.
The present modest choice of parameters ($N_{\rm max} = 4010$) produce a
{\tt common} block of about 4~Mb.

Before compiling, check the {\ZZ {FORTRAN}} directives in the {\tt Makefile},
\ie whether {\tt f77} or {\tt g77} and use the highest optimization level.
Compile by the command `{\tt make nbody6}' which should produce the
executable {\tt nbody6}.
One possible reason for failure could be the {\tt CPU} timer function
{\tt etime} in routine {\tt cputim.f} which is system dependent.
Any other complaints and strange run-time behaviour should be reported after
making every effort to ascertain the problem.
For frequent usage it is recommended to create separate working directories.
It is also a good idea to save the original files when making changes.

To start a test run, place the executable and a template input file
{\tt input} in one directory and type the command
`{\tt nbody6 $<$ input $>$ output \&}'.
Although the results are machine- and compiler-dependent, it is expected that
the run will finish normally but there is always a chance that some difficult
configuration may occur.

The restart facility can be tested as follows.
First specify option \#~1 = 1 as input.
This will ensure a {\tt common} dump on {\tt fort.1} if the {\tt CPU} time is
exceeded or {\tt TIME $>=$ TCRIT}, with {\tt TCRIT} given in the input file.
The calculation can then be continued from the {\tt common} save by
`{\tt nbody6 $<$ rs $>$} output2 \&',
using the restart input file `{\tt KSTART TCOMP}'.
Here {\tt KSTART = 1} denotes a new run (followed by the required input)
or $= 2$ for standard restart, and {\tt TCOMP} is the {\tt CPU} time in
minutes.
Also note the possibility of reading some new parameters at restart if
{\tt KSTART $>$ 2} (see routine {\tt modify.f} and the restart
file {\tt rs}).
Before restarting in the same directory, any important output files in
capitals ({\it e.g.} {\tt ESC, OUT9}) must be renamed and/or deleted
(likewise any fort.82 and fort.83).

The code includes an optional provision for automatic error checking.
Thus if option \#~2 = 2, any output interval with relative energy error
$\vert \Delta E / E \vert > 5 \,Q_{\rm E}$, with $Q_{\rm E}$ the tolerance,
is restarted from the previous time with reduced values of the basic
integration parameters $\etaI$ and $\etaR$.
If instead the energy error lies in the interval
[$Q_{\rm E},\, 5\,Q_{\rm E}$], the accuracy parameters are reduced by an
appropriate amount.
Likewise, an increase (up to the initial values) is carried out for relative
errors below $Q_{\rm E} / 5$ (see routine {\tt check.f}).
Note the use of two options (with {\tt common} saves on {\tt fort.1} and
{\tt fort.2}) in order to employ the energy check independently of
terminations at arbitrary times.
Consequently, in the case of a halted calculation, the previous good
{\tt common} save must be copied from {\tt fort.2} to {\tt fort.1} before 
the restart (possibly with modified accuracy parameters).

\section{Input parameters}

A standard input file consists essentially of six lines defining the
membership, accuracy and decision-making parameters, options, KS
integration, IMF power-law data and virial theorem scaling.
Some of these quantities are given in Table~\ref{variables} and
routine {\tt define.f} contains a complete listing.
Many are dimensionless while others have an astrophysical meaning
(see Aarseth 2001a for the neighbour scheme in {\ZZ {NBODY2}}).
Likewise, the options {\tt KZ(J)} can be selected by consulting
Table~\ref{options}.
Hence only a few values need to be changed in the input template once a
calculation has been performed.
Input files for primordial binaries ({\tt inbins}) and 3D tidal field
({\tt inlog}) are included in the directory {\tt Docs}.
Uploaded binaries with option 22 can be regularized initially by
specifying {\ZZ {NBIN0}}.

For most purposes, and a wide range of particle numbers, the standard choice
of the accuracy parameters $\etaI, \etaR$ and $\etaU$ should suffice.
Having decided on the membership $N$, a maximum neighbour number of
$n_{\rm max} \simeq 2 N^{1/2}$ is usually adequate in the absence of
primordial binaries.
One good strategy for choosing the output intervals is to adopt
$\Delta t_{\rm adj} = 2.0$ for energy checks and escaper removal and take
$\Delta t_{\rm out} = 10.0$ for main output and data analysis.
The choice of the relative energy tolerance $Q_{\rm E}$ is a matter of
taste.
%of taste and a slightly larger value than quoted in Table~\ref{variables}
%may be acceptable.

\section{Initial conditions}

Different choices of initial conditions are available.
These include the Plummer model, two orbiting clusters or a massive
bound/unbound  binary ({\tt KZ(5) = 1, 2, 4}).
Several types of IMF (Salpeter or Kroupa, Tout \& Gilmore 1993) may be
generated by option \#~20.
However, it is envisaged that the user will provide original initial
conditions for new investigations
($m_i, {\bf r}_i, {\bf v}_i, ~i = 1, 2, .., N$).
One possibility is to use the stand-alone code {\tt aking6.f} and input
{\tt inking} for generating King models [Heggie \& Ramamani 1995]
(no warranty!) included in the directory {\tt Docs}.
A package for producing initially rotating cluster models is also available
[Spurzem \& Einsel 2002, private communication].

The specially generated initial distribution is read from {\tt fort.10} with
option \#~22~=~2 before scaling to standard internal units of total energy
$-0.25$ (for bound clusters) and $\sum m_i = 1$.
Primordial binaries and single stars can be uploaded using \#~22~=~4 instead.
Unless special precaution is taken, the same scaling is applied generally.
Thus in scaled units and overall virial equilibrium the mean square velocity
is $<v^2> = 1/2$ and the crossing time (denoted {\tt TCR}) is
$t_{\rm cr} = 2 \sqrt 2$, independent of $N$.

With the initial length unit {\tt RBAR} specified in pc and the mean mass
{\tt ZMBAR} (which is not preserved with option \#~20 $>=$ 0) in
$M_{\odot}$, any internal quantity may be converted to astrophysical values.
A convenient set of conversion factors is given by
{\tt RBAR, SMU, VSTAR, TSTAR} for distances (pc), masses
($M_{\odot}$), velocities (${\rm km~sec}^{-1}$) and times (in Myr).
Binary periods in years or days can be obtained from the $N$-body units
using {\tt YRS} or {\tt DAYS} in the standard Keplerian expression without
$2 \pi$.
Moreover, length scales may also readily be converted to solar radii
({\tt SU}) or astronomical units ({\tt AU}).

\section{Decision-making}

Although individual time-steps are used in the general integration, the 
Hermite scheme employs discrete block-steps which enables many particles to
be advanced in tandem.
Consequently the main integration cycle comprises a small number of calls
to the next level of routines, with a few other important tasks performed
either before or at the end.
This structure facilitates the investigation of any strange behaviour such
as infinite looping since the offending routine can be identified, whereupon
the principle of bisecting can often be applied to locate the problem.

Essentially the integration cycle consists of the following procedures:

\begin{itemize}
\item Determine the next block of particles due for advancement
\item Update any close encounter solutions (KS and/or chain)
\item Predict coordinates and velocities (neighbours or full $N$)
\item Advance the solution for each body (neighbour force or total force)
\item Deal with any close encounter terminations (KS, chain or mergers)
\item Check stellar evolution (updating of radii or mass loss)
\end{itemize}

The strategy for the first step relies on the simple device of determining
the smallest value of $t + \Delta t_i$ for the members in the block, denoted
by $t_{\rm new}$, where $t$ is the current time.
The corresponding time-steps are obtained by a relative force criterion of
the form
\be
\Delta t_i = 
\left( \frac  {\eta {\bf \vert F }_i \vert}
 {{\ddot {\bf \vert F }}_i \vert} \right)^{1/2} \,,
\ee
where $\eta$ is a small dimensionless constant ensuring convergence.
To fit in with the more efficient block-step scheme, the `natural' step
is truncated (or quantized) to the nearest factor 2 value commensurate with
{\tt TIME}.
This requirement implies that {\tt mod(TIME,$\Delta t_i)=0$} for
both the old and new time-step at the updating of a Hermite solution.
A more sensitive expression is obtained by including the other force
derivatives.
Given a list of particles that are due for treatment in some larger
interval, the actual block-step members are determined by identifying
those with $T_0 + \Delta t_j = t_{\rm new}$, where $T_0$ is the time
of the last neighbour (so-called irregular) force evaluation.

The basic integration cycle is given by the following lines of code:

\begin{verbatim}
*       Advance the pointer (<= NXTLEN) and select next particle index.
   50 LI = LI + 1
      IF (LI.GT.NXTLEN) GO TO 1
      I = NXTLST(LI)
      TIME = T0(I) + STEP(I)
*
*       See whether the regular force needs to be updated (IR > 0).
      IF (T0R(I) + STEPR(I).LE.TIME) THEN
          IR = 1
      ELSE
          IR = 0
      END IF
*
*       Advance the irregular step.
      CALL NBINT(I,IKS,IR,XI,XIDOT)
*
*       See whether the regular step is due.
      IF (IR.GT.0) THEN
          CALL REGINT(I,XI,XIDOT)
      END IF
*
*       Determine next block time (note STEP may shrink in REGINT).
      TMIN = MIN(TNEW(I),TMIN)
\end{verbatim}

As discussed in subsequent sections, the special case of KS and chain
regularization merits special treatment.
Following the relevant predictions, new solutions of the irregular and
(if required) regular force polynomials are obtained for each particle
in turn, together with coordinate and velocity corrections.
After all the members on the current block have been advanced, procedures
relating to any terminations are carried out.
This may necessitate an occasional return to the main routine for the
specified task, defined by the flow control indicator
(\cf Table~\ref{control}).
The last step of the main cycle deals with the optional stellar evolution,
examined at small quantized intervals of $\le 10^2$~yr.

\section{Data management} \label{data}

It is the purpose of a code to produce results.
However, there is a wide choice of quantities that can be constructed from
the basic particle data.
Here a dual-purpose strategy has been adopted and if this is not convenient,
special data analysis can readily be included either during integration or
at output times (\ie called from routine {\tt adjust.f}).

The data structure itself forms the backbone of the code and will be
summarized first (also see book p.117).
Given a number of KS solutions $N_{\rm p}$, the particle arrays for the
components of each pair $I_{\rm p}$ are placed in locations $2 I_{\rm p}-1$
and $2 I_{\rm p}$, with the corresponding centre of mass (c.m.) in
$N + I_{\rm p}$.
Hence all KS pairs appear contiguously in the order in which they are
initialized.
This scheme facilitates a sequential treatment of all the particles, with
force summations and neighbour lists referring to increasing locations in
the range [$2 N_{\rm p}+1, \, N + N_{\rm p}$].
It entails relabelling particle references in neighbour or perturber lists
after each new or terminated KS pair but is a small cost for preserving
simplicity.
The structure of a given list for particle $i$ is of the form
{\tt LIST(k,i)}, where $k=1$ is used for the neighbour number.
The neighbour strategy in {\ZZ {NBODY2}} is similar [Aarseth 2001a].
In the case of force summations involving neighbours with $j > N$, the \cm is
replaced by the components (which must be obtained by a KS transformation) if
the \cm approximation is not satisfied.

In order to introduce more complex systems, such as stable triples or
temporary chain subsystems, the device of `ghost' particles of zero mass is
used (see below).
This facilitates recovery of the original state without affecting the overall
data structure.
Pointers to the individual members are saved and any relevant information may
be obtained as required, although the identification of multiple hierarchies
is non-trivial (see Appendix~C of book).

Two types of output are produced.
Important results are presented in file names using capitals, while
supplementary diagnostics appear in {\tt fort.n} with $n > 2$.
The main files provide the following information in mathematical notation:

\begin{itemize}
\item {\tt ESC}~~~~~~~~$t\,({\rm Myr}),\, m \,(M_{\odot}) ,\, v^2/<v^2>,\, v\,$
$(\kms),\, k^{\ast},\,{\cal N}_i$ for each escaper
\item {\tt OUT9}~~~~~~$E_{\rm b},\, e,\, E_{\rm cm},\, r,\, m_k,\, m_l,\, P \,({\rm days}),$
$\,{\cal N}_k,\, {\cal N}_l,\, k_k^{\ast},\,k_l^{\ast}$ for each KS binary
\item {\tt HIARCH}~~~$t,\,a_0,\,a_1,\,e_1,\,a_1(1-e_1),\,P_1/P_0,\,m_{\rm bin}/m_3,\,$
$R_{\rm pcrit}, \,m_{\rm bin},\,m_1/m_2,\,{\cal N}_{1,2,3}$
\item {\tt HIDAT}~~~~${\cal N}_k,\,{\cal N}_l,\,{\cal N}_m,\,k^{\ast},\,m_1,\,m_2,$
$\,m_3,\,r,\,e_{\rm max},\,e_0,\,e_1,\,P_0,\,P_1$ for hierarchies
\item {\tt OUT3}~~~~~~$m,\,x,\,y,\,z,\,{\dot x},\,{\dot y},\,{\dot z},\,{\cal N}_i$
for $i = [1,\,N + N_{\rm p}]$\, (binary format)
\end{itemize}

Here ${\cal N}_i$ represent original particle names, $E_{\rm cm}$ is the
specific \cm binding energy, $P$ denotes the period (days), $k^{\ast}$ is the
stellar type, $a_1 (1 - e_1)$ is the outer pericentre and $R_{\rm pcrit}$
the corresponding stability boundary.
The maximum eccentricity in the Kozai cycle is given by $e_{\rm max}$, while
$e_0$ and $e_1$ refer to the respective eccentricities.
Note that the file {\tt HIARCH} contains information on the formation and
termination of hierarchical systems whereas {\tt HIDAT} provides a summary
at each main output.
All the above are optional (\cf the multi-valued \#~8) and the file
{\tt OUT3} is the data bank which may be produced at main output with
specified frequency.
Because the binary format is currently saved in single precision, subsequent
reading and analysis of this data must be consistent.
Note the use of {\it two} records for each output which enables arrays of
variable length to be read.

A variety of results also appear as standard output.
This takes the form of an error check at intervals ${\Delta t}_{\rm adj}$,
(with ${\tt DE = \Delta E/E}$) when the density centre is updated and
escapers removed.
Considerably more information is given at intervals ${\Delta t}_{\rm out}$
which are best kept commensurate with ${\Delta t}_{\rm adj}$.
In this connection, note the facility (\#~32) to increase the output
intervals if the energy binding the cluster changes by factors of 2.

We summarize some of the most important quantities given at main output.
This information is organized in distinct (optional) groups.
The first line gives the particle number, average neighbour number
($<{\tt NNB}>$), KS solutions ({\tt KS}), number of mergers ({\tt NM} and
{\tt MM}, standard and higher order) and single stars ({\tt NS}).
Among useful quantities describing the cluster evolution are the half-mass
radius ($<{\tt R>}$), tidal radius ({\tt RTIDE}), core radius ({\tt RC}) and
membership ({\tt NC}), energy in binaries and mergers
(${\tt EB/E}$ and ${\tt EM/E}$) and the time in Myr ({\tt T6}).
Escapers are removed outside {\tt 2*RTIDE} with option \#~23.

If \#~8 is active, there is a summary of original and exchanged binaries, the
average and maximum eccentricity ($<{\tt E>}$ and {\tt EMAX}), as well as
distributions of stellar population types and binary binding energies.
The energy budget is also summarized.
For historical reasons, the energy binding the cluster is saved in
{\tt E(3)}, while {\tt E(1)} and {\tt E(2)} give the energy of primordial
and dynamically formed binaries.
For the definition of the total energy see Eq.~(9.29) of the book.
A large number of counters are displayed; see Table~\ref{counts} for a list
of the most important.
In large calculations, some time-step counters may exceed the integer limit
and are reset to zero above $2 \times 10^9$.

Finally, note the optional time offset provision (\#35) which prevents large
values of {\tt TIME/STEP}.
Thus if the limit {\tt DTOFF} is exceeded, {\tt TIME} is reset to zero and
{\tt TOFF} is used to obtain the total time {\tt TTOT}.
This is only of practical relevance when printing the current value of the
time as {\tt TIME + TOFF}.

The question of adding extra variables to the {\tt common} blocks often
arises.
Here we mention two useful strategies.
The simplest case is to introduce a new labelled {\tt common} in the header
file {\tt common6.h} so that these variables become available in most
routines.
At the same time, the size of the {\tt common} save must be increased
accordingly (\cf routine {\tt mydump.f}) and the whole code recompiled.
A slight drawback is that old {\tt common} saves cannot be read in the
usual way.
The alternative is to create dummy variables in an existing labelled
{\tt common} which can be used for newly created variables without affecting
the overall size.
However, the latter method needs to anticipate future demands.

\section{Two-body regularization}

The KS method first appeared in a standard $N$-body code in time for the
Cambridge IAU Colloquium \#~10 [Aarseth 1972].
It was an instant success and has proved a mainstay for treating binaries
and close two-body encounters ever since.
Like any versatile algorithm, it has progressed through several distinct
versions until ending up with the highly accurate Stumpff formulation
[Mikkola \& Aarseth 1998].
Although the underlying mathematics is very precise, its implementation is
something of a black art, employing a variety of heuristic procedures.
This is particularly the case when dealing with more complicated interactions
in compact subsystems.

An arbitrary number of KS solutions are treated at the same time.
Decision-making is essentially controlled by two input parameters which are
modified at each output if option \#~16 is active.
A search for close encounter candidates is carried out if the time-step
becomes suitably small; \ie ${\Delta t}_i < {\Delta t}_{\rm cl}$,
subject to the distance test $R < R_{\rm cl}$ in order to ensure dominant
two-body motion (see book for definitions).
If necessary, the actual regularization is delayed until both the components
are advanced to the same time.
Note that the relative time-step criterion gives very similar values for
different masses during close encounters.
Implementations of the KS transformations and polynomial initialization have
been described in considerable detail elsewhere (Aarseth 1985, 2001b, 2003).
It is therefore sufficient to discuss some aspects of the
decision-making.\footnote{The data structure is described in
section~\ref{data}.}

As noted in a previous section, all relevant KS solutions are considered
at the start of the integration cycle and not advanced beyond the end of the
block-step.
For this purpose, the regularized time-step is converted to physical units
(book Eq.~(11.1)).
In practice most primordial binaries are unperturbed, with time intervals
often exceeding the typical block-step, and an efficient sorting and insert
procedure is employed (see book p.143).
The exceptional case of a physical collision is treated differently and
requires a new block-step to be created (for a discussion of time
quantization see book Eq.~(12.16)).
An algorithm for specifying intervals of unperturbed motion is also given in
the book.

Termination is essentially controlled by the relative perturbation, except
that soft binaries and hyperbolic flybys are also subject to a distance test
in terms of the initial separation.
The strategy for terminating a strongly perturbed hard binary depends on the
suitability of a switch to chain regularization.
The ideal case is that the latter treatment may be adopted but there are many
situations when such configurations are not sufficiently compact (see below).
Hence we may instead have repeated switching of particle pairs according to
dominant two-body solutions, which is less satisfactory.

Following a KS termination, new polynomials and time-steps are assigned to
the components.
All relevant neighbour and perturber lists must also be updated in order to
be consistent with the new particle sequence.
However, the latter task deals with integer arithmetic and is quite fast.
Since the \cm time-step is inevitably small for strong perturbations, the new
steps may be comparable and hence do not result in smaller block-steps.
In any case, small values may increase rapidly if the situation permits
(\ie doubling every other step).
Finally, we emphasize that the use of regularization places a lower limit on
the Hermite time-steps, thereby saving significant computational efforts.

\section{Hierarchical systems}

The presence of binaries often lead to the formation of long-lived
hierarchies.
The computational requirements for direct integration of the inner binary may
be quite severe; yet the semi-major axis hardly changes even for modest
distance ratios.
Since this is the most important binary element, it would seem justified to
adopt the \cm approximation and recast the KS solution for the outer
component, thereby increasing the period and replacing one direct
integration.
This procedure neglects short-term fluctuations and assumes no secular
change, in qualitative agreement with first-order perturbation theory.

Although simple stability criteria were already introduced in the mid 1980s,
the chaos approach provides greater theoretical justification and has given
rise to a semi-analytical criterion [Mardling \& Aarseth 1999].
More recently, a general three-body stability criterion has been developed
from first principles [Mardling 2008].
It is valid for a wide mass range and all inclinations and the robustness
has been verified by numerical tests.
We also mention that the code now includes an averaging method
to model the eccentricity oscillations (Kozai cycles) induced by high
inclinations.

The identification of suitable configurations is carried out for small \cm
time-steps and the first part of the procedure employs the same algorithm
as for chain regularization (see below).
Subsequently, the division of labour is essentially made by comparing the
outer pericentre, $R_{\rm p} = a_1 (1 - e_1)$, with the inner semi-major
axis, $a_0$.
Roughly speaking, the case $R_{\rm p} > 3 a_0$ justifies a search for stable
systems.
Further conditions, such as the requirement of the outer binary being hard
and not too strongly perturbed must also be satisfied in addition to the
stability test.

A new hierarchy is initialized in a similar way as for standard KS.
However, when examining the data structure we distinguish between the outer
component being a single particle or binary.
In the case of a triple, the outer body is defined as a ghost particle after
combining it with the inner \cm into a new wider KS pair.
However, with a second binary, both the KS solution and associated \cm are
made inactive by prescribing large values of $T_0$ and setting the relevant
masses to zero.
Here the latter plays the role of the ghost particle for triples and once
the \cm location $j > N$ has been identified from the ghost name, the pair
index is simply given by $I_{\rm p} = j - N$.
For stability purposes, triples may be considered as degenerate quadruples
and a small correction term is included for the smallest binary.
Although rare, higher-order systems also occur and are treated in an
analogous manner (their data structure is discussed in the book). 

Further stability checks are made at each apocentre passage because the outer
orbit may have changed due to perturbations.
Mass loss from the inner binary reduces the spacing ratio and necessitates a
stability test which is performed {\it in situ}.
Following termination of a triple, the inner binary is initialized as a KS
solution while the outer component already has the standard form (but is not
in the correct location).
In the case of a quadruple, the second KS pair and ghost \cm are initialized
{\it in situ} in the usual way after the neighbour list is formed.
The force discontinuity arising from the changed configurations is handled in
several ways with appropriate differential corrections to the energy budget.
Note that the initialization and reconstruction of hierarchies employ the
small labelled {\tt common} block {\ZZ {BINARY}} which contains original
masses and particle names (as well as the basic KS variables for any second
pair).

A new stability criterion has now been developed from first principles
[Mardling 2008].
This formulation includes the effect of the inner eccentricity as well as the
inclination and is currently (Nov 2007) valid for
$m_2/m_1 > 0.05$ or $m_3/m_1 > 0.05$.
Although the stability boundaries are qualitatively similar, the new
criterion is more general.
The old criterion is still used in a few places, with the inclination now
given in radians.

\section{Chain regularization}

Several methods are available for treating strong interactions involving more
than two particles.
The original AZ three-body regularization was implemented first, followed by
Heggie's global method for four particles [\cf Aarseth 1985].
For simplicity, these formulations did not include external perturbations.
The versatile chain regularization [Mikkola \& Aarseth 1990, 1993, 1996]
which includes perturbations has proved more effective.
Because of the increased complexity, the relevant routines are placed in the
separate directories {\tt Chain} and {\tt Nchain}.
Since only one configuration at a time is considered at present, the two
unperturbed treatments are still maintained but rarely needed.

Close multiple encounters are characterized by one or more small \cm
time-steps.
It is therefore convenient to perform the main decision-making for initiating
multiple regularization at apocentre (defined as the turning point of radial
velocity when $R > a$).
In the case of a single particle approaching a hard binary, the condition
for acceptance depends on the separation as well as the impact parameter.
Thus too long delay leads to KS termination by large perturbation, while
wide separations may be inefficient (unless for ultra-compact
configurations).
Algorithms for initialization and integration of chain subsystems are
described in the book, together with specific procedures for changing the
membership (reduction or increase).
Such interactions tend to be short lived.
Note the importance of including various stability tests (see above) since
the ejection of a particle may otherwise produce a long-lived hierarchical
system which can be studied more efficiently in another way.

Termination usually occurs when the third body escapes or two surviving
binaries become well separated and are more suitable for KS treatment.
The initialization of one or two KS solutions is carried out at termination
rather than in the usual way during integration.
A more difficult situation arises when the most distant particle is still
bound to the subsystem.
It then becomes a question of whether to retain such a member as part of
the chain or include its effect as a strong perturber.
In this connection, note that the system size is used for perturber
selection and also that the perturbers are predicted at every function
evaluation (\ie many times per step) by the Bulirsch--Stoer [1966]
integrator.
Both the close encounter separation and the characteristic gravitational
radius (defined by the mass products and total energy) are used to limit
the system size.

The data structure of the basic chain algorithm employs quantities expressed
in the local \cm frame.
Transformations to global values must therefore be made before including the
effect of perturbers, and likewise at termination.
Since the device of ghost particles is adopted to preserve the sequential
arrangement, the actual masses are saved together with the corresponding
names.
It is then a simple matter to recover the relevant global locations for
initialization.
In the case of three initial members, the binary components are placed in the
first two single particle locations, as for standard KS termination, while
the third body is turned into a ghost particle {\it in situ}.
The termination of two binaries, on the other hand, yields the initial
members in the first four locations.
However, this arrangement cannot be assumed to persist since a new KS may be
formed during the intervening interval and in any case, a four-body system
may be reduced by escape.
Another point worth noting is that the particle assigned as the \cm may in
fact escape, in which case a new reference body is determined.
We also mention that the current scheme now caters for up to six members,
although more than four is quite rare.

The treatment of the chain \cm requires special care.
Thus the force and its first derivative are first obtained in the usual way,
whereupon differential corrections are added.
This entails subtracting the standard c.m. contributions from any perturbers
and adding the respective individual mass-weighted terms.
For consistency, similar corrections are carried out when dealing with the
perturbers.
Again the overheads of checking the neighbour lists for identification is
modest compared with the actual function evaluations.
Moreover, note that the accumulated duration of all the chain
regularizations only represents a small fraction of the total time.

Some comments on the time management may be helpful.
The choice of intervals for the internal integration is based on the
principle of convergence for coordinate prediction.
Essentially the maximum interval for advancing the solutions is determined
by examining $T_0 + {\Delta t}_j$ for the relevant perturbers as well as the
\cm itself.
Typically, the Bulirsch--Stoer time-steps are somewhat smaller than this
interval.
An inversion from physical to regularized time usually ensures that the
maximum is barely exceeded (see book for the algorithm).
Additional topics, such as slow-down and quantization of time are also
discussed extensively in the book.

\section{Stellar evolution}

For greater realism, the code includes several options for mass loss and
finite-size effects.
The treatment of stellar evolution is based on fast look-up functions
which provide information on the stellar type, radius as well as core mass
for a given initial mass, age and metallicity [Tout \etal 1997,
Hurley \etal 2000].
The current position in the HR diagram is checked at frequent intervals
determined by the evolution rate and the remaining time of each
characteristic stage.
A list of stellar evolution indicators $k^{\ast}$ is included in routine
{\tt define.f} and two look-up times (\ie previous and next value) are used
for decision-making.
Reimers-type wind loss is adopted in addition to supernova events, where the
latter result in neutron star or even black hole formation.

For convenience, mass loss corrections are implemented when the accumulated
wind loss exceeds $1\%$.
This entails modifying the force and first derivative of each neighbour, as
well as subtracting the potential energy change assuming instantaneous mass
loss from the cluster.
In the case of KS solutions, the orbit is expanded at constant eccentricity
together with force updating of the \cm neighbours.
Neutron stars are currently assigned a kick velocity sampled from a
Maxwellian with relatively large dispersion [\cf Hansen \& Phinney 1997]
which usually leads to disruption of close binaries and escape of single
stars.
However, there is also a choice of modest kick velocities.
An optional procedure (\#~37) is included in order to avoid a sudden close
approach by high-velocity particles.

The code now contains a realistic recipe for physical collisions,
implemented for KS solutions and all multiple regularizations.
This scheme has been used successfully in the private {\ZZ {NBODY4}}.
Depending on stellar type of the components, complete mixing or common
envelope evolution is adopted together with mass loss.
In the case of a KS pair, the new \cm body is initialized as a
single particle and the second component turned into a massless escaper.
With chain regularization, an iterative procedure is used to determine the
exact pericentre if the closest two-body separation lies inside a suitably
small value.
The energy of the collision pair is evaluated indirectly using well defined
variables rather than from the two-body elements.
With more than three chain members, the membership is reduced and the
calculation continued, otherwise standard termination follows.

Finally, we mention optional tidal circularization (\#~27).
If \#~27 = 1, a sequential procedure is employed
[Portegies Zwart \etal 1997].
In this case, discrete changes of the orbital elements $a, e$ are made.
Relevant modifications of the KS variables are carried out if
$a (1 - e) < 4 r^{\ast}_1$, with $r^{\ast}_1$ the largest radius (see book
for detailed algorithm).
Alternatively, continuous circularization is adopted if \#~27 = 2
[Mardling \& Aarseth 2001].
Several adjustments are usually made because the stellar radii tend to
increase with time.
Note that for very large eccentricities, angular momentum conservation
gives rise to considerable shrinkage.
One way to reach large eccentricity is for the inner binary to experience
favourable Kozai oscillations.

\section{External fields}

The code includes two types of external tidal field which will be described
(for details see chapter 8 of book).
Linearized equations are appropriate for nearly circular orbits with
small vertical displacements and are therefore suitable for simulating
open clusters.
Two optional variants are available: (i) \#~14 = 1 for the standard case
based on Oort's constants, and (ii) \#~14 = 2 which applies to a galactic
point mass (also linearized).

The relevant tidal terms (converted from the length unit {\tt RBAR} and
total mass) are initialized in routine {\tt xtrnl0.f} and the perturbations
are added in {\tt xtrnlf.f} and also in {\tt xtrnlp.f} for KS.
The additional terms in the equations of motion are simple and enables
explicit force derivatives to be employed for the Hermite integration.
The corresponding contributions to the total energy are included
(\cf {\tt ETIDE} in {\tt xtrnlv.f}), thereby facilitating conservation.
However, the linearized form of the equations of motion are not appropriate
well outside the tidal radius.
Hence a more detailed exploration of ejected stars is restricted to modest
distances.

In order to broaden the scope for studying more general cluster motion, a
full 3D galactic model has been implemented (\#~14~= 3).
Note that the external effect is only included in {\tt xtrnlf.f} for the
direct integration, consistent with treating binaries in the c.m.
approximation here.
We adopt a composite galaxy model consisting of three components:
(i) central point mass, (ii) Miyamoto--Nagai (1975) disk, and (iii)
logarithmic potential.
This model gives a good representation of the Galaxy and satisfies the
requirement of being computationally convenient for the Hermite scheme.
In order to provide more flexibility, any combination of the components may
be used by varying some of the input parameters (\cf routine {\tt xtrnl0.f}
and the special input template in directory {\tt Docs}).

The tidal effect on cluster members is obtained by including the
differential force of the Galaxy with respect to the cluster centre.
This necessitates integrating the cluster guiding centre as a point mass
orbiting the Galaxy (routines {\tt gcinit.f} and {\tt gcint.f}), using the
total force functions.
Consequently, the cluster motion is known to high accuracy throughout
the calculation.
As before, distant stars may be considered as escapers and are removed if
\#~23~$>$~0.
Unless specified initially, a value {\tt RTIDE} = 50~pc is adopted for
the tidal radius.
Alternatively, the ejected members may be added to a test population forming
the tidal tail and integrated by a fast method (see below).

The equations of motion for the full galactic potential do not admit an
energy integral as in the linearized case.
This raises the question of an alternative way of accounting for the tidal
energy change.
From general principles, this contribution is provided by the integral of
$\dot w = {\bf v} \cdot {\bf P}$, where ${\bf P}$ is the tidal force.
Likewise, the terms for ${\ddot w}$ are available using the galactic force
derivative.
This enables a Taylor series solution to be obtained at the end of each
regular step.
The third-order terms are incomplete and their inclusion give worse results,
so are omitted.
The individual mass-weighted terms are accumulated (\cf {\tt ETIDE} in
{\tt regint.f}) and included in the expression for the total energy.
Note that small output intervals (i.e. $\Delta t_{\rm adj} < 1$) may
introduce apparent spurious energy errors here.
So far, experience with this scheme has been favourable.

It may also be of interest to study a star cluster embedded in gas,
with the possibility of including time-dependent decay.
A Plummer sphere coinciding with the centre of mass has been adopted as
an optional feature (\#~14~= 3 or 4) which can be employed independently.
Three extra input parameters are required (\cf routine {\tt xtrnl0.f} and
{\tt define.f}), namely the total mass and length scale (saved as square)
as well as the decay time for gas expulsion [Kroupa \etal 2001], all in
$N$-body units. Note that energy conservation does not apply if the
Plummer mass decays (hint: set large energy tolerance {\tt QE}).

The equations of motion are modified analogous to the case of linear
external perturbation by including the force components in the regular
polynomial.
Likewise, the relative perturbations for KS solutions (force and its
derivative) are evaluated in simplified form assuming the same softened
central distance.
The escape criterion is modified to include the background potential
and likewise the total energy correction.
Another aspect needing attention is mass loss due to evolving stars
(routines {\tt ficorr.f} and {\tt fcorr.f}).
Again the correction procedure is similar to the standard case
(\#~14~= 1).

Since the standard definitions of crossing time and virial ratio are
not suitable in this formulation, alternative expressions are used.
The former is now given by $2 R_{\rm h}/V$, with $R_{\rm h}$ the
half-mass radius and $V$ the rms velocity.
A consistent expression for the latter can be derived from first principles
(based on summing $m {\bf r} \cdot {\bf F}$) which leads to combining the
potential and virial energies in the denominator.
Finally, total energy conservation is achieved by adding the sum of tidal
energy contributions in the usual way at output time (\cf {\tt ETIDE} in
routine {\tt adjust.f}).

\section{Tidal tails}

Procedures for studying the growth of tidal tails by fast integration have
been added to the code.
This facility can be particularly useful for cluster orbits in the 3D
galactic potential since the tidal energy corrections tend to become
less accurate at large distances.
Consequently, a substantial speed-up may be achieved for large systems.
The basic idea of the scheme is to employ the standard variables, with
the tidal tail particles saved in unused parts of the arrays (which
necessitates making a suitable declaration of maximum array size).
In fact, the standard size of the {\tt common} blocks remains unchanged,
with four counters or pointers replacing part of a redundant integer array.
There are only three new routines of modest size and another five are
modified slightly.

Each tidal tail member is initialized for integration at the time of
escaper removal, i.e. outside a distance {\tt 2*RTIDE} where {\tt RTIDE}
is specified (in $N$-body units) at input if non-zero.
Data for the first member is saved in location
{\tt ITAIL0 = NZERO + {\rm min}(KMAX,NBIN0+10)}, with {\tt NBIN0} the
initial number of primordial binaries.
This is safely above the largest value used for direct integration.
Routine {\tt tail0.f} increases the membership and copies the current
coordinates and velocities to appropriate locations, expressed with
respect to the Galactic Centre.
New time-steps are then assigned after obtaining the force and first
derivative, whereupon the integration variables are initialized in the
usual way.
The membership is denoted by {\tt NTAIL} and any loop over the whole
population is made from {\tt I = ITAIL0} to
{\tt NTTOT = ITAIL0 + NTAIL - 1}.
Thus all stars in the tidal tail would require array sizes
(i.e. {\tt NMAX}) of at most {\tt 2*NZERO + KMAX}, depending on the number
of primordial binaries.

Since the integration is done in the Galactic frame, the corresponding
force and first derivative are obtained by the usual expressions without
differential correction (routine {\tt xtrnlt.f}).
This approach assumes a full galactic model (i.e.~\#~14~=~3) but a
linearized tidal field can readily be included if local coordinates are
used instead.
The integration itself is performed by routine {\tt ntint.f}, called from
{\tt intgrt.f} at the end of each block-step for any member satisfying
the usual condition $T_0 + \Delta t \le t$.
The algoritm is a simplified version of {\tt nbint.f}, with standard
prediction added.
Note that the time-steps are usually of the maximum size, currently 1.0
time units and quantized values are used for a uniform treatment.
This ensures that the data are available with the highest accuracy at main
output times.

Some simple additions to other routines facilitate the decision-making.
Although no new options are required, the tidal tail integration is
initiated with option \#~23 $\ge 3$ and \#~14~=~3.
Any relevant treatment is therefore executed for an existing population,
i.e. {\tt NTAIL} $> 0$.
Finally, option \#~3 is used to control the output of results for data
analysis and plotting.
A value of 1 gives rise to the standard output file {\tt OUT3} in binary
format.
Various multiple choices are available, with \#~3 = 4 producing a formatted
file {\tt OUT34} in astrophysical units (pc and $\kms$) which contain all
the stars with respect to the density centre ({\tt ASCI} format), while
\#~3~=~5 also yields {\tt OUT3}.
A small header contains both memberships and the time in Myr.

\section{Numerical problems}

A code is never fully tested and the range of initial conditions may be
much wider than has been considered so far.
The question of validity is not an easy one, and there are also accuracy
issues in spite of careful treatment.

Consider the simple case of an intermediate-mass cluster
(say $N \simeq 500$) with realistic mass spectrum and the recommended
maximum neighbour number, $2 N^{1/2}$.
Thus a massive wide KS binary may introduce significant errors because a
perturber list containing all the neighbours would be too small; \ie
$a (1 + e)/ \gamma_{\rm min}^{1/3} > R_{\rm s}$ (with $R_{\rm s}$ the
neighbour radius) and yet the binding energy may be large.
This situation improves with increasing $N$ since we have
$R_{\rm cl} \propto 1/N$ while the interparticle separation goes as
$1/N^{1/3}$.

Another difficult condition arises for borderline cases of hierarchies which
may be long-lived but not accepted as stable.
If it occurs, frequent switching of solutions also leads to inefficiency as
well as loss of accuracy.
It goes without saying that one cannot anticipate future problems since so
much depends on the type of investigation.
However, in general, the numerical task gets harder during advanced stages of
evolution when the central density is smaller and more complex hierarchical
configurations are formed.
In order to investigate a given problem, it is useful to halt the
calculation before including some diagnostics.
This can be achieved by typing `{\tt touch STOP}' at an arbitrary time,
whereupon a {\tt common} save will occur on {\tt fort.1}, from which a
restart can be made.

Although the external energy binding the cluster may be feeble, superhard
binaries tend to have strong interactions enhanced by their gravitational
focusing.
Even if not present initially, such binaries may form via tidal
circularization.
In any case, some shrinkage by dynamical means is possible before ejection
by recoil intervenes.
Thus a small actual deviation of the total energy may give rise to a
disproportionately large relative error because the external energy appears
in the denominator.
In this connection, note that the accumulated energy change is
given at the end of a calculation.

Finally, a word of warning concerning output time intervals.
Thus it is highly desirable to employ quantized values like 0.125 or 1.0
instead of intervals not commensurate with 1. In fact, the latter could
lead to code crash.

\section{Acknowledgements}

The following colleagues have made substantial contributions to the code:

\begin{itemize}

\item Seppo Mikkola ~~~~~~~~~~~~~~~~~~~Two-body and chain regularization
\item Chris Tout \& Jarrod Hurley ~~Stellar evolution and collisions
\item Rosemary Mardling~~~~~~~~~~~~~~Stability of hierarchical systems
\end{itemize}

\bigskip
Version 7.2.0 11/2008.

\newpage
\section*{References}

\medskip
\noindent
Aarseth, S.J. [1972], `Direct integration methods for the $N$-body problem',
in {\it Gravitational $N$-Body Problem} ed. M. Lecar (D. Reidel), 373--87.

\medskip
\noindent
Aarseth, S.J. [1985], `Direct methods for $N$-body simulations',
in {\it Multiple Time Scales} ed. J.U. Brackbill \& B.I. Cohen
(Academic Press), 377--418.

\medskip
\noindent
Aarseth, S.J. [1999], `From NBODY1 to NBODY6: the growth of an industry',
{\it PASP} {\bf 111}, 1333--46.

\medskip
\noindent
Aarseth, S.J. [2001a], `NBODY2: a direct $N$-body integration code',
{\it New Astron.} {\bf 6}, 277--91.

\medskip
\noindent
Aarseth, S.J. [2001b], `Regularization methods for the $N$-body problem',
in {\it The Restless Universe}, ed. B.A. Steves \& A.J. Maciejewski
(Inst. Phys. Publ.), 93--108.

\medskip
\noindent
Aarseth, S.J. [2003], {\it Gravitational N-Body Simulations} (Cambridge University Press).

\medskip
\noindent
Aarseth, S.J. \& Zare, K. [1974], `A regularization of the three-body problem',
{\it Celes. Mech.} {\bf 10}, 185--205.

\medskip
\noindent
Ahmad, A. \& Cohen, L. [1973], `A numerical integration scheme for the $N$-body
gravitational problem', {\it J. Comput. Phys.} {\bf 12}, 389--402.

\medskip
\noindent
Bulirsch, R. \& Stoer, J. [1966], `Numerical treatment of ordinary differential
equations by extrapolation methods',
{\it Num. Math.} {\bf 8}, 1--13.

\medskip
\noindent
Eggleton, P.P., Fitchett, M.J. \& Tout, C.A. [1989], `The distribution of visual
binaries with two bright components',
\APJ {\bf 347}, 998-1012. (Also see Errata in \APJ {\bf 354}, 387.)

\medskip
\noindent
Hansen, B.M.S. \& Phinney, E.S. [1997], `The pulsar kick velocity distribution',
\MN {\bf 291}, 569--77.

\medskip
\noindent
Heggie, D.C. [1974], `A global regularisation of the gravitational N-body problem',
{\it Celes. Mech.} {\bf 10}, 217--41.

\medskip
\noindent
Heggie, D.C. \& Ramamani, N. [1995], `Approximate self-consistent models for
tidally truncated star clusters',
\MN {\bf 272}, 317--22.

\medskip
\noindent
Hurley, J.R., Pols, O.R. \& Tout, C.A. [2000], `Comprehensive analytical formulae
for stellar evolution as a function of mass and metallicity',
\MN {\bf 315}, 543--69.

\medskip
\noindent
Kustaanheimo, P. \& Stiefel, E. [1965], `Perturbation theory of Kepler motion
based on spinor regularization',
{\it J. Reine Angew. Math.} {\bf 218}, 204--19.

\medskip
\noindent
Kroupa, P., Tout, C.A. \& Gilmore, G. [1993], `The distribution of low-mass stars
in the Galactic disc',
\MN {\bf 262}, 545--87.

\medskip
\noindent
Kroupa, P., Aarseth, S.J. \& Hurley, J. [2001], `The formation of a bound star
cluster: from the Orion Cluster to the Pleiades', \MN {\bf 321}, 699--712.

\medskip
\noindent
Makino, J. [1991], `Optimal order and time-step criterion for Aarseth-type
$N$-body integrators', \APJ {\bf 369}, 200--12.

\medskip
\noindent
Makino, J. \& Aarseth, S.J. [1992], `On a Hermite integrator with Ahmad--Cohen
scheme for gravitational many-body problems',
{\it Publ. Astron. Soc. Japan} {\bf 44}, 141--51.

\medskip
\noindent
Mardling, R.A. [2008], `A general three-body stability criterion', \MN {\bf xxx}, yyy--zzz.

\medskip
\noindent
Mardling, R.A. \& Aarseth, S.J. [1999], `Dynamics and stability of three-body systems',
in {\it The Dynamics of Small Bodies in the Solar System}, ed. B.A. Steves \& A. Roy
(Kluwer), 385--92.

\medskip
\noindent
Mardling, R.A. \& Aarseth, S.J. [2001], `Tidal interactions in star cluster simulations',
\MN {\bf 321}, 398--420.

\medskip
\noindent
Mikkola, S. [1985], `A practical and regular formulation of the $N$-body equations',
\MN {\bf 215}, 171--7.

\medskip
\noindent
Mikkola, S. and Aarseth, S.J. [1990], `A chain regularization method for the
few-body problem',
\CMD {\bf 47}, 375--90.

\medskip
\noindent
Mikkola, S. \& Aarseth, S.J. [1993], `An implementation of $N$-body chain
\hbox{regularization',}
\CMD {\bf 57}, 439--59.

\medskip
\noindent
Mikkola, S. \& Aarseth, S.J. [1996], `A slow-down treatment for close binaries',
\CMD {\bf 64}, 197--208.

\medskip
\noindent
Mikkola, S. \& Aarseth, S.J. [1998], `An efficient integration method for
binaries in $N$-body simulations',
{\it New Astron.} {\bf 3}, 309--20.

\medskip
\noindent
Miyamoto, M. \& Nagai, R. [1975], `Three-dimensional models for the distribution
of mass in galaxies', {\it Publ. Astron. Soc. Japan} {\bf 27}, 533--43.

\medskip
\noindent
Portegies Zwart, S.F., Hut, P., McMillan, S.L.W. \& Verbunt, F. [1997],
`Star cluster ecology, II. Binary evolution with single-star encounters',
{\it Astron. Astrophys.} {\bf 328}, 143--57.

\medskip
\noindent
Spurzem, R., Baumgardt, H. \& Ibold, N. [2003], `A parallel implementation of an
$N$-body integrator on general and special-purpose computers',
\MN in press.

\medskip
\noindent
Tout, C.A., Aarseth, S.J., Pols, O. \& Eggleton, P. [1997], `Rapid binary
star evolution for $N$-body simulations and population synthesis',
\MN {\bf 291}, 732--48.

\newpage

\section{Appendix}

In this Appendix we provide some tables of useful information.
Table~\ref{params} defines the parameters used in the general {\tt common}
block, together with representative values for a test calculation with 1000
single particles and 1000 primordial binaries. Note the provision of extra
KS solutions in case of additional close encounters in the early stages.

\begin{table}[h]
\centering
\caption{{\it {\ZZ {FORTRAN}} parameters.}}
\label{params}
\begin{tabular}{rrlr}
\hline\hline
$N_{\rm max}$ &{\ZZ{NMAX}} &Total particle number and \cm bodies &4010 \\
$K_{\rm max}$ &{\ZZ{KMAX}} &KS solutions &1010 \\
$L_{\rm max}$ &{\ZZ{LMAX}} &Neighbour limit &100 \\
$M_{\rm max}$ &{\ZZ{MMAX}} &Hierarchical binaries &10 \\
$M_{\rm dis}$ &{\ZZ{MLD}} &Recently disrupted KS components &22 \\
$M_{\rm reg}$ &{\ZZ{MLR}} &Recently regularized KS components &22 \\
$M_{\rm high}$ &{\ZZ{MLV}} &High-velocity particles &10 \\
$M_{\rm cloud}$ &{\ZZ{MCL}} &Interstellar clouds &10 \\
$N_{\rm chain}$ &{\ZZ{NCMAX}} &Chain membership &10 \\
\hline\hline
\end {tabular}
\end{table}

\bigskip
\bigskip
Table~\ref{variables} contains an example of standard input parameters and
typical values for an $N = 1000$ test run (a larger value of {\ZZ {NNBMAX}}
may be used for primordial binaries).
Both the book and {\ZZ {FORTRAN}} notations are given for convenience.

\begin{table}[h]
\centering
\caption{{\it Integration parameters.}}
\label{variables}
\begin{tabular}{rrlr@{.}l}
\hline\hline
$\etaI$ &{\ZZ{ETAI}} &Time-step parameter for irregular force &0&02 \\
$\etaR$ &{\ZZ{ETAR}} &Time-step parameter for regular force &0&03 \\
$S_0$ &{\ZZ{RS0}}&Initial radius of the neighbour sphere &0&3 \\
$n_{\rm max}$ &{\ZZ{NNBMAX}} &Maximum neighbour number &70&0 \\
$\Delta t_{\rm adj}$ &{\ZZ{DTADJ}} &Time interval for energy check &2&0 \\
$\Delta t_{\rm out}$ &{\ZZ{DELTAT}} &Time interval for main output &10&0 \\
$Q_{\rm E}$ &{\ZZ{QE}} &Tolerance for energy check &1&0 $\times 10^{-5}$\\
$R_{\rm V}$ &{\ZZ{RBAR}} &Virial cluster radius in pc &2&0 \\
$M_{\rm S}$ &{\ZZ{ZMBAR}} &Mean stellar mass in solar units (\#20=0) &0&8 \\
$Q_{\rm vir}$ &{\ZZ{Q}} &Virial theorem ratio ($T/\vert U - 2 W\vert $) &0&5 \\
$\Delta t_{\rm cl}$ &{\ZZ{DTMIN}} &Time-step criterion for close encounters
                    &4&0 $\times 10^{-5}$\\
$R_{\rm cl}$ &{\ZZ{RMIN}} &Distance criterion for KS regularization &0&001\\
$\etaU$ &{\ZZ{ETAU}} &Regularized time-step parameter &0&2 \\
$h_{\rm hard}$ &{\ZZ{ECLOSE}} &Energy per unit mass for hard binary &1&0 \\
$\gamma_{\rm min}$ &{\ZZ{GMIN}} &Limit for unperturbed KS motion &1&0 $\times 10^{-6}$\\
$\gamma_{\rm max}$ &{\ZZ{GMAX}} &Termination criterion for soft binaries &0&01 \\
\hline\hline
\end {tabular}
\end{table}

\newpage
The main options are listed below.
For a complete list see routine {\tt define.f}.
To find where option \# $J$ is used, type `{\tt grep KZ(J) *.f}' in
the $N$-body directories (or just 'KZ' in the directory {\tt Chain}).

\begin{table}[h]
\centering
\caption{{\it Optional features.}}
\label{options}
\begin{tabular}{rll}
\hline\hline
1 &Common save on unit 1 by {\tt touch STOP} or TIME $>$ TCRIT \\
2 &Common save on unit 2 at output time or restart \\
3 &Data bank on unit 3 with specified frequency \\
5 &Standard initial conditions (=0: uniform; =1: Plummer) \\
6 &Output of significant \& KS binaries (=1, 2, 3 \& 4) \\
7 &Output of Lagrangian radii (several types) \\
8 &Primordial binaries (extra input required) \\
10 &Regularization diagnostics (=2: NEW KS \& END KS) \\
12 &HR diagnostics of evolving stars (interval {\ZZ{DTPLOT}}) \\
13 &Interstellar clouds (extra input required) \\
14 &External tidal force; open or globular clusters \\
15 &Multiple regularization or hierarchical systems \\
16 &Updating of regularization parameters $R_{\rm cl},\,\Delta t_{\rm cl}$ \\
17 &Modification of $\etaI$ and $\etaR$ by tolerance $Q_{\rm E}$ \\
18 &Primordial triples (extra input required) \\
19 &Synthetic stellar evolution with mass loss ($>$=3) \\
20 &Different types of initial mass functions (=0: standard) \\
21 &Extra output line ({\ZZ{MODEL\, \#,\, CPU,\, DMIN,\, AMIN,\, RMAX)}} \\
22 &Initial conditions $m_i, {\bf r}_i, \dot{\bf r}_i$ on unit \#10 (=2, -1) \\
23 &Removal of distant escapers (isolated or tidal) \\
26 &Slow-down of KS and/or chain regularization (=1, 2, 3) \\
27 &Tidal circularization (=1: sequential; =2: chaos; =3: GR capture) \\
28 &GR radiation for NS \& BH binaries (with \#19 = 3; choice of \#27) \\
30 &Chain regularization ($>$ 1: special diagnostics) \\
32 &Increase of output interval (limited by $t_{\rm cr}$) \\
33 &Distribution of block-steps at output (1 or 2) \\
34 &Roche-lobe overflow ($< $2: synhronization) \\
35 &Integration time offset (standard = 100 time units) \\
36 &Step reduction for hierarchies (not recommended!) \\
37 &Neighbour additions ($> 0$: high-velocity; $> 1$: not rec.) \\
38 &Force polynomial corrections (=0: I $>$ N; not rec.) \\
40 &Neighbour number control ($>=2$: fine-tuning to {\ZZ{NNBMAX}}/5) \\
\hline\hline
\end{tabular}
\end{table}

\newpage

Table~\ref{counts} defines significant counters, together with an actual
example from run with 1800 single particles and 200 primordial binaries.
%Table~\ref{control} lists the control indicators used for decision-making.

\begin{table}[h]
\centering
\caption{{\it Characteristic counters.}}
\label{counts}
\begin{tabular}{lll}
\hline\hline
Name &
\multicolumn{1}{c}{Definition} &
\multicolumn{1}{c}{Counts} \\
\hline\hline
{\ZZ {NSTEPI}} &Irregular time-steps &$8.0\times10^7$ \\
{\ZZ {NSTEPR}} &Regular time-steps &$2.1\times 10^7$ \\
{\ZZ {NBLOCK}} &Block steps &$5.1\times 10^6$ \\
{\ZZ {NKSTRY}} &Regularization attempts &$2.4\times 10^6$ \\
{\ZZ {NKSREG}} &KS regularizations &$2.2\times 10^3$ \\
{\ZZ {NKSHYP}} &Hyperbolic regularizations & \multicolumn{1}{c}{800} \\
{\ZZ {NKSMOD}} &KS slow-down modifications &$6.6\times 10^4$ \\
{\ZZ {NKSPER}} &Unperturbed two-body orbits &$4.6\times 10^{11}$ \\
{\ZZ {NMERGE}} &Hierarchical mergers & \multicolumn{1}{c}{359} \\
{\ZZ {NEWHI}}  &Independent new hierarchies & \multicolumn{1}{c}{25} \\
{\ZZ {NCHAIN}} &Chain regularizations & \multicolumn{1}{c}{86} \\
{\ZZ {NSTEPU}} &Regularized time-steps &$4.2\times 10^7$ \\
{\ZZ {NSTEPC}} &Chain integration steps &$1.1\times 10^5$ \\
{\ZZ {NMDOT}}  &Stellar evolution look-ups &$5.0\times 10^4$ \\
{\ZZ {NSN}}    &Supernova events & \multicolumn{1}{c}{5} \\
{\ZZ {NWD}}    &White dwarfs & \multicolumn{1}{c}{90} \\
{\ZZ {NCOLL}}  &Stellar collisions & \multicolumn{1}{c}{4} \\
{\ZZ {NBS}}    &Blue stragglers & \multicolumn{1}{c}{2} \\
{\ZZ {NSYNC}}  &Circularized binaries & \multicolumn{1}{c}{9} \\
{\ZZ {NSESC}}  &Single escapers & \multicolumn{1}{c}{1833} \\
{\ZZ {NBESC}}  &Binary escapers & \multicolumn{1}{c}{176} \\
{\ZZ {NMESC}}  &Hierarchical escapers & \multicolumn{1}{c}{2} \\
\hline\hline
\end {tabular}
\end{table}

\bigskip
Table~\ref{control} lists the control indicators used for decision-making.

\begin{table}[b]
\vspace{-8cm}
\centering
\caption{{\it Indicator for flow control.}}
\label{control}
\begin{tabular}{rll}
\hline\hline
$0$ &Standard value \\
$1$ &New KS regularization \\
$2$ &KS termination \\
$3$ &Output and energy check \\
$4$ &Three-body regularization \\
$5$ &Four-body regularization \\
$6$ &New hierarchical system \\
$7$ &Termination of hierarchy \\
$8$ &Chain regularization \\
$9$ &Physical collision \\
$-1$ &Exceptional cases \\
\hline\hline
\end{tabular}
\end{table}

\end{document}
